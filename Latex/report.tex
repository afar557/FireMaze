\documentclass{article}

\setlength{\headsep}{0.75 in}
\setlength{\parindent}{0 in}
\setlength{\parskip}{0.1 in}

%=====================================================
% Add PACKAGES Here (You typically would not need to):
%=====================================================

\usepackage[margin=1in]{geometry}
\usepackage{amsmath,amsthm}
\usepackage{fancyhdr}
\usepackage{enumitem}
\usepackage{graphicx}
%=====================================================
% Ignore This Part (But Do NOT Delete It:)
%=====================================================

\theoremstyle{definition}
\newtheorem{problem}{Problem}
\newtheorem*{fun}{Fun with Algorithms}
\newtheorem*{challenge}{Challenge Yourself}
\def\fline{\rule{0.75\linewidth}{0.5pt}}
\newcommand{\finishline}{\vspace{-15pt}\begin{center}\fline\end{center}}
\newtheorem*{solution*}{Solution}
\newenvironment{solution}{\begin{solution*}}{{\finishline} \end{solution*}}
\newcommand{\grade}[1]{\hfill{\textbf{($\mathbf{#1}$ points)}}}
\newcommand{\thisdate}{\today}
\newcommand{\thissemester}{\textbf{Rutgers: Spring 2021}}
\newcommand{\thiscourse}{CS 440: Introduction to Artificial Intelligence} 
\newcommand{\thishomework}{Number} 
\newcommand{\thisname}{Name} 

\newcommand{\thisheading}{
   \noindent
   \begin{center}
   \framebox{
      \vbox{\vspace{2mm}
    \hbox to 6.28in { \textbf{\thiscourse \hfill \thissemester} }
       \vspace{4mm}
       \hbox to 6.28in { {\Large \hfill Project \#\thishomework \hfill} }
       \vspace{2mm}
         \hbox to 6.28in { { \hfill \thisdate \hfill} }
       \vspace{2mm}
       \hbox to 6.28in { \emph{Names: \thisname \hfill }}
      \vspace{2mm}}
      }
   \end{center}
   \bigskip
}

%=====================================================
% Some useful MACROS (you can define your own in the same exact way also)
%=====================================================


\newcommand{\ceil}[1]{{\left\lceil{#1}\right\rceil}}
\newcommand{\floor}[1]{{\left\lfloor{#1}\right\rfloor}}
\newcommand{\prob}[1]{\Pr\paren{#1}}
\newcommand{\expect}[1]{\Exp\bracket{#1}}
\newcommand{\var}[1]{\textnormal{Var}\bracket{#1}}
\newcommand{\set}[1]{\ensuremath{\left\{ #1 \right\}}}
\newcommand{\poly}{\mbox{\rm poly}}


%=====================================================
% Fill Out This Part With Your Own Information:
%=====================================================


\renewcommand{\thishomework}{1: This Maze is on \emph{Fire}} %Homework number
\renewcommand{\thisname}{Aamna Farooq, Nada Elshamaa, and Asma Makhdoom} % Your name
 \graphicspath{ {./images/} }

\begin{document}

\thisheading



\begin{problem}
	Write an algorithm for generating a maze with a given dimension and obstacle density p.
	\begin{enumerate}[label=(\Alph*)]
		\item Algorithm $A$ divides an instance of size $n$ into $3$ subproblems of size $n/2$ each, recursively solves each one, and then takes $O(n)$ time 
		to combine the solutions and output the answer.
		\item Algorithm $B$ divides an instance of size $n$ into $2$ subproblems, one with size $n/2$ and one with size $n/3$, recursively solves each one, and then takes $O(n)$ time 
		to combine the solutions and output the answer. 
		\item Algorithm $C$ divides an instance of size $n$ into $4$ subproblems of size $n/5$ each, recursively solves each one, and then takes $O(n^2)$ time 
		to combine the solutions and output the answer. 
		\item Algorithm $D$ divides an instance of size $n$ in to $2$ subproblems of size $n-1$ each, recursively solves each one, and then takes $O(1)$ time to combine the solutions and output the answer. 
	\end{enumerate}
	
	For each algorithm, write a recurrence that captures its runtime and \emph{use the recursion tree method} to solve this recurrence and find the \emph{tightest asymptotic} upper bound on runtime of the algorithm. \grade{25} 
\end{problem}
\begin{solution}
	The solution to problem one goes here. 
	\begin{enumerate}[label=(\Alph*)]
		\item Recurrence for algorithm $A$:  
		
%=====================================================
% LaTeX Tip: The figure environment is used for displaying the figure. Inside that, we use \includegraphics for adding a different file for each of the figure.
% What you need to do is to create your figure using any tool you like (including drawing it by hand and scanning it), turn it into a pdf, name it "tree[X].pdf" 
% where you replace [X] by A,B,C, or D depending on which part you are solving (i.e., use treeA.pdf, treeB.pdf,...). Finally, copy the file in the same directory
% as this template and you should see that getting compiled into your solution. You can change the width=0.5\textwidth command to fix the size of the figure
% in the final PDF. Finally, if the figure appears in the next page, do not worry about it as they will still have the proper caption. 
%=====================================================		

		\begin{figure}[h!]
			\centering
			\IfFileExists{treeA.pdf}{\includegraphics[width=0.5\textwidth]{treeA.pdf}}{No Figure Yet}
		\caption{Recursion tree for algorithm $A$.} 
		\end{figure}
		
		\item Recurrence for algorithm $B$:  
		
		\begin{figure}[h!]
			\centering
			\IfFileExists{treeB.pdf}{\includegraphics[width=0.5\textwidth]{treeB.pdf}}{No Figure Yet}
		\caption{Recursion tree for algorithm $B$.} 
		\end{figure}
		
		\item Recurrence for algorithm $C$:  
		
		\begin{figure}[h!]
			\centering
			\IfFileExists{treeC.pdf}{\includegraphics[width=0.5\textwidth]{treeC.pdf}}{No Figure Yet}
		\caption{Recursion tree for algorithm $C$.} 
		\end{figure}
		
		\item Recurrence for algorithm $D$:  
		
		\begin{figure}[h!]
			\centering
			\IfFileExists{treeD.pdf}{\includegraphics[width=0.5\textwidth]{treeD.pdf}}{No Figure Yet}
		\caption{Recursion tree for algorithm $D$.} 
		\end{figure}
		
	\end{enumerate}
\end{solution}
\smallskip

\begin{problem}
	You are given a permutation of $\set{1,\ldots,n}$ in an array $A[1:n]$. Design a \emph{randomized} algorithm that finds an index of some \emph{even} number 
	in this array in only $O(1)$ \emph{expected runtime}. You can assume that 
	creating a random number from $1$ to $n$ can be done in $O(1)$ time. Note that the array $A$ is already in the memory and your algorithm does not need to `read' this array.  \grade{25}
	
	\emph{Example:} Suppose the input is $[3,2,4,5,1]$; then the algorithm can return either index $2$ or index $3$. 	
\end{problem}

\smallskip

\begin{solution}
	Solution to problem two goes here. 
\end{solution}


\smallskip


\begin{problem}
	Write $BFS$ and $A*$ algorithms (using the euclidean distance metric) that take a maze and determine the shortest path from $S$ to $G$ if one exists. For as large a dimension as your system can handle, generate a plot of the average $'number$ $of$ $nodes$ $explored$ $by$ $BFS$ - $number$ $of$ $nodes$ $explored$ $by$ $A*$' vs $'obstacle$ $density$ $p'$. If there is no path from $S$ to $G$, what should this difference be? 
	
\end{problem}

\smallskip

\begin{solution}
    If there was no path from S to G then there would be no difference between the number of nodes explored. This is because A* and BFS each would try to exhaust every possible option and would therefore have the same number of nodes explored.
    
	\begin{figure}[h]
	\includegraphics[width=12cm]{dim1000,10avg,step01}
	\caption{Plot using a dimension of 1000, average of 10, 0<p<1 with step = 0.01}
	\centering
	\end{figure}
	
\end{solution}

\smallskip

\begin{problem}\label{DP}
	Alice and Bob are delivering for the Pizza store that has $n$ orders today. From past experience, they both know that if Alice delivers the $i$-th order, the tip would be $A[i]$ dollars,
	and if Bob delivers, the tip would be $B[i]$ dollars. Alice and Bob are close friends and their goal is to maximize the total tip they can get and then distribute it evenly between themselves. The problem is that 
	Alice can only handle $a$ of the orders and Bob can only handle $b$ orders. They do not know how to do this so they ask for your help. 
	
	Design a dynamic programming algorithm that given the arrays $A[1:n],B[1:n]$, and integers $1 \leq a,b \leq n$, finds the largest amount of tips that Alice and Bob can collect together subject to their constraints. The 
	runtime of your algorithm should be $O(n \cdot a \cdot b)$ in the worst case. \grade{25}
	
	\emph{Example:} Suppose $A = [2,4,5,1]$, $B=[1,7,2,5]$, $a=2$ and $b=1$. Then the answer is $14$ by Alice delivering $2,3$ and Bob delivering $4$ (so $4+5+5 = 14$), or Alice delivering $1,3$ and Bob delivering $2$ ($2+5+7=14$).  
\end{problem}

\smallskip

\begin{solution}
	Solution to problem four goes here. 
\end{solution}

\smallskip



\begin{challenge}
	A standard dynamic programming algorithm for Problem~\ref{DP} requires $O(n \cdot a \cdot b)$ \emph{space} in addition to the same runtime. The goal of this question is to improve
	this space bound and further extend the algorithm.  \grade{0}
	\begin{enumerate}[label=(\alph*)]
	\item Design an algorithm for this problem that requires only $O(n+a \cdot b)$ space instead and still has the worst-case runtime of $O(n \cdot a \cdot b)$ (this part is rather standard and is not that challenging). 
	\item Design an algorithm with the same space $O(n+a \cdot b)$ and time $O(n \cdot a \cdot b)$ that in addition to the value of the largest possible tip, can also output the set of deliveries Alice and Bob should take
	in order to achieve such a collection of tips (this part may be really challenging).
	\end{enumerate}
	
\end{challenge}

\begin{fun}
Recall that Fibonacci numbers form a sequence $F_n$ where $F_0 = 0$, $F_1 = 1$, and $F_n = F_{n-1} + F_{n-2}$. 
The standard algorithm for finding the $n$-th Fibonacci number takes $O(n)$ time. The goal of this question is to design a significantly faster algorithm for this problem. \grade{0}
	\begin{enumerate}[label=(\alph*)]
		\item Prove by induction that for all $n \geq 1$: 
		\begin{align*}
			\begin{bmatrix}
				1 & 1 \\
				1 & 0
			\end{bmatrix}^n
			&= 	\begin{bmatrix}
				F_{n+1} & F_n \\
				F_n & F_{n-1}
			\end{bmatrix}.
		\end{align*}
		\item Use the first part to design an algorithm that finds $F_n$ in $O(\log{n})$ time. 
	\end{enumerate}	 
\end{fun}
\end{document}